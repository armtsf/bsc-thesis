% !TeX root=../main.tex
% در این فایل، عنوان پایان‌نامه، مشخصات خود، متن تقدیمی‌، ستایش، سپاس‌گزاری و چکیده پایان‌نامه را به فارسی، وارد کنید.
% توجه داشته باشید که جدول حاوی مشخصات پروژه/پایان‌نامه/رساله و همچنین، مشخصات داخل آن، به طور خودکار، درج می‌شود.
%%%%%%%%%%%%%%%%%%%%%%%%%%%%%%%%%%%%
% دانشگاه خود را وارد کنید
\university{دانشگاه تهران}
% پردیس دانشگاهی خود را اگر نیاز است وارد کنید (مثال: فنی، علوم پایه، علوم انسانی و ...)
\college{پردیس دانشکده‌های فنی}
% دانشکده، آموزشکده و یا پژوهشکده  خود را وارد کنید
\faculty{دانشکدهٔ مهندسی کامپیوتر}
% گروه آموزشی خود را وارد کنید (در صورت نیاز)
\department{گروه معماری کامپیوتر}
% رشته تحصیلی خود را وارد کنید
\subject{مهندسی کامپیوتر}
% گرایش خود را وارد کنید
\field{معماری سیستم‌های کامپیوتری}
% عنوان پایان‌نامه را وارد کنید
\title{پیاده‌سازی قراردادهای هوشمند بر بستر زنجیره‌ی بلوکی با رویکرد بهبود مقیاس‌پذیری}
% نام استاد(ان) راهنما را وارد کنید
\firstsupervisor{دکتر سعید صفری}
\firstsupervisorrank{دانشیار}
%\secondsupervisor{}
%\secondsupervisorrank{}
% نام استاد(دان) مشاور را وارد کنید. چنانچه استاد مشاور ندارید، دستورات پایین را غیرفعال کنید.
%\firstadvisor{دکتر مشاور اول}
%\firstadvisorrank{استادیار}
%\secondadvisor{دکتر مشاور دوم}
% نام داوران داخلی و خارجی خود را وارد نمایید.
\internaljudge{دکتر سیامک محمدی}
\internaljudgerank{دانشیار}
%\externaljudge{دکتر داور خارجی}
%\externaljudgerank{دانشیار}
%\externaljudgeuniversity{دانشگاه داور خارجی}
% نام نماینده کمیته تحصیلات تکمیلی در دانشکده \ گروه
%\graduatedeputy{دکتر نماینده}
%\graduatedeputyrank{دانشیار}
% نام دانشجو را وارد کنید
\name{آرمیتا}
% نام خانوادگی دانشجو را وارد کنید
\surname{صفا}
% شماره دانشجویی دانشجو را وارد کنید
\studentID{۸۱۰۱۹۴۳۵۰}
% تاریخ پایان‌نامه را وارد کنید
\thesisdate{بهمن ۱۳۹۸}
% به صورت پیش‌فرض برای پایان‌نامه‌های کارشناسی تا دکترا به ترتیب از عبارات «پروژه»، «پایان‌نامه» و «رساله» استفاده می‌شود؛ اگر  نمی‌پسندید هر عنوانی را که مایلید در دستور زیر قرار داده و آنرا از حالت توضیح خارج کنید.
%\projectLabel{پایان‌نامه}

% به صورت پیش‌فرض برای عناوین مقاطع تحصیلی کارشناسی تا دکترا به ترتیب از عبارت «کارشناسی»، «کارشناسی ارشد» و «دکتری» استفاده می‌شود؛ اگر نمی‌پسندید هر عنوانی را که مایلید در دستور زیر قرار داده و آنرا از حالت توضیح خارج کنید.
%\degree{}
%%%%%%%%%%%%%%%%%%%%%%%%%%%%%%%%%%%%%%%%%%%%%%%%%%%%
%% پایان‌نامه خود را تقدیم کنید! %%
%\dedication
%{
%	{\Large تقدیم به:}\\
%	\begin{flushleft}{
%			\huge
%			همسر و فرزندانم\\
%			\vspace{7mm}
%			و\\
%			\vspace{7mm}
%			پدر و مادرم
%		}
%	\end{flushleft}
%}
%% متن قدردانی %%
%% ترجیحا با توجه به ذوق و سلیقه خود متن قدردانی را تغییر دهید.
%\acknowledgement{
%	سپاس خداوندگار حکیم را که با لطف بی‌کران خود، آدمی را به زیور عقل آراست.
%	
%	در آغاز وظیفه‌  خود  می‌دانم از زحمات بی‌دریغ اساتید  راهنمای خود،  جناب آقای دکتر ... و ...، صمیمانه تشکر و  قدردانی کنم که در طول انجام این پایان‌نامه با نهایت صبوری همواره راهنما و مشوق من بودند و قطعاً بدون راهنمایی‌های ارزنده‌ ایشان، این مجموعه به انجام نمی‌رسید.
%	
%	از جناب آقای دکتر ... که  زحمت مشاوره‌، بازبینی و تصحیح این پایان‌نامه را تقبل فرمودند کمال امتنان را دارم.
%	
%	%از همکاری و مساعدت‌های دکتر ... مسئول تحصیلات تکمیلی و سایر کارکنان دانشکده بویژه سرکار خانم ... کمال تشکر را دارم.
%	
%	با سپاس بی‌دریغ خدمت دوستان گران‌مایه‌ام، خانم‌ها ... و آقایان ... در آزمایشگاه ...، که با همفکری مرا صمیمانه و مشفقانه یاری داده‌اند.
%	
%	و در پایان، بوسه می‌زنم بر دستان خداوندگاران مهر و مهربانی، پدر و مادر عزیزم و بعد از خدا، ستایش می‌کنم وجود مقدس‌شان را و تشکر می‌کنم از خانواده عزیزم به پاس عاطفه سرشار و گرمای امیدبخش وجودشان، که بهترین پشتیبان من بودند.
%}
%%%%%%%%%%%%%%%%%%%%%%%%%%%%%%%%%%%%
%چکیده پایان‌نامه را وارد کنید
\fa-abstract{
امروزه زنجیره‌ی بلوکی یکی از تکنولوژی‌های در حال پیشرفت است که کاربردهای گوناگون آن باعث فراگیرشدن آن شده‌است. از مهم‌ترین کاربردهای آن می‌توان به قراردادهای هوشمند اشاره کرد. بـا تـوجـه بـه حجـم بسیار بـالاي تـراکنش‌های مـوجـود و اسـتفاده‌ی زنجیره‌‌ی بـلوکی از یک شـبکه‌ی هـمتابـه‌هـمتا، چـالـش مقیاس‌پـذیري را می‌تـوان بـه عـنوان یکی از بـزرگ‌تـرین چـالـش‌هـا در نـظر گـرفـت. در حـال حـاضـر بسیاري از زنجیره‌هـای بـلوکی مـطرح، حجـمی بیشتر از ۱۰۰ گیگابـایت را شـامـل می‌شـونـد. هـنگامی که یک گـره کاونـده بـه شـبکه‌ی زنجیره‌ی بـلوکی اضـافـه می‌شـود، ابـتدا تـمامی زنجیره‌ی قبلی را بـاربـرداری می‌کند و سـپس شـروع بـه حـل فـرآیند پیچیده‌ی محاسـباتی می‌کند. این حجـم بسیار بـالا که بـه طـور روزمـره نیز بر مـقدار آن افـزوده می‌شـود، یک چـالـش اسـاسی هـنگام اسـتفاده از زنجیره‌ی بلوکی است. همین موضوع مانعی مهم در برابر گسترش برنامه‌های مبتنی بر زنجیره‌ی بلوکی و استفاده‌ی بیشتر از این فناوری است. برای مقابله با مشکل مقیاس‌پذیری، راه‌حل‌های متفاوتی ارائه شده است که برخی از این راه‌حل‌ها با رویکرد تغییر ماهیت زنجیره‌ی بلوکی و عناصر سازنده‌ی آن مطرح می‌شوند. دسته‌ای دیگر از راه‌حل‌ها بر نحوه‌ی نگهداری اطلاعات تمرکز دارند. از آن‌جایی که در قراردادهای هوشمند داده‌ی ذخیره‌شده در زنجیره‌ی بلوکی بایت‌کد مربوط به منطق برنامه است، می‌توان از فشرده‌سازی به عنوان یک راهکار کاهش حجم اطلاعات زنجیره‌ی بلوکی استفاده کرد. در این پژوهش با اعمال روش‌های مختلف فشرده‌سازی بر یک زنجیره‌ی بلوکی پیاده‌سازی شده و بررسی این روش‌ها از نظر میزان فضای صرفه‌جویی‌شده، سربار زمانی و توان مصرفی، مقایسه‌ای بین الگوریتم‌های فشرده‌سازی متفاوت انجام خواهیم داد. 
}
% کلمات کلیدی پایان‌نامه را وارد کنید
\keywords{زنجیره‌ی بلوکی، قرارداد هوشمند، مقیاس‌پذیری، فشرده‌سازی، اتریوم}
% انتهای وارد کردن فیلد‌ها
%%%%%%%%%%%%%%%%%%%%%%%%%%%%%%%%%%%%%%%%%%%%%%%%%%%%%%