% !TeX root=../main.tex
\chapter{مفاهیم اولیه و پیش‌زمینه}
%\thispagestyle{empty} 

%%%%%%%%%%%%%%%%%%%%% INTRO %%%%%%%%%%%%%%%%%%%%%
\section{مقدمه}
در این فصل برای روشن‌شدن مفاهیم و اصطلاحات حول موضوع این پژوهش، به تعریف برخی مفاهیم اساسی که پیش‌نیاز درک این پروژه هستند می‌پردازیم. هم‌چنین پیشینه‌ی موضوع و پژوهش‌های مرتبط که سابقاً انجام شده‌اند را بررسی خواهیم کرد.

%%%%%%%%%%%%%%%%%%%%% DEFINITIONS %%%%%%%%%%%%%%%%%%%%%
\section{مفاهیم اولیه}
\subsection{زنجیره‌ی بلوکی}
اولین زنجیره‌ی بلوکی توسط یک فرد یا گروه ناشناس به نام ساتوشی ناکاموتو
\LTRfootnote{Satoshi Nakamoto}
 در سال ۲۰۰۸ معرفی شد. 
 \cite{nakamoto2012bitcoin}
 این طرح در سال بعد توسط ناکاموتو به عنوان هسته‌ای اصلی رمزارز بیت‌کوین پیاده‌سازی شد. زنجیره‌ی بلوکی به عنوان دفترکل
 \LTRfootnote{ledger}
  برای تمامی تراکنش‌های روی شبکه‌ی رمزارزی فعالیت می‌کند. در اوت ۲۰۱۴ اندازه‌ی زنجیره‌ی بلوکی بیت‌کوین به ۲۰گیگابایت رسید. این مقدار با رشد سریع، در ژانویه‌ی ۲۰۱۷ تا ۱۰۰گیگابایت رشد کرد. 
  \cite{inbook}
  
  
زنجیره‌ی بلوکی یک لیست از بلوک‌هاست که با رمزنگاری به هم متصل شده‌اند. هر بلوک یک درهم‌ساز
\LTRfootnote{hash}، یک برچسب زمانی
\LTRfootnote{timestamp}
و اطلاعات مربوط به تراکنش از بلوک پیشین را نگهداری می‌کند. \cite{10.5555/2994437}
از آن‌جایی که زنجیره‌ی بلوکی یک شبکه‌ی توزیع‌شده و غیرمتمرکز است، امکان این که تراکنش‌های ثبت‌شده بدون تغییر دادن تمامی بلوک‌های بعدی ویرایش شوند وجود ندارد. به این ترتیب گره‌های فعال در شبکه می‌توانند تراکنش‌ها را به طور مستقل درستی‌سنجی نمایند.


به طور کلی از زنجیره‌ی بلوکی در زمینه‌های زیر استفاده می‌شود:
\begin{itemize}
	\item ارز دیجیتال
	\item قراردادهای هوشمند
	\LTRfootnote{smart contracts}
	\item ثبت و نگهداری سوابق
	\item اینترنت اشیا
\end{itemize}
در ادامه قراردادهای هوشمند که تمرکز اصلی این پژوهش هستند را معرفی می‌کنیم.
\subsubsection{قراردادهای هوشمند}
ایده‌ی قراردادهای هوشمند ابتدا توسط نیک سابو
\LTRfootnote{Nick Szabo}
 مطرح شد.سپس با ظهور فناوری زنجیره‌ی بلوکی، این ایده عملی شد. به این ترتیب و با پیاده‌سازی قراردادهای هوشمند بر بستر زنجیره‌ی بلوکی، اصطلاح قراردادهای هوشمند به صورت هر نوع محاسبات عام‌منظوره‌ای که روی زنجیره‌ی بلوکی یا دفترکل توزیع‌شده‌ای انجام شود تعریف می‌شود. با این تعریف و با درنظرگرفتن مثال‌هایی از استفاده‌ی آن، مانند اتریوم
 \LTRfootnote{Ethereum}
 ، می‌توان گفت که قرارداد هوشمند مفهوم سنتی قرارداد را ندارد و به هر نوع برنامه‌ی کامپیوتری گفته می‌شود. قراردادهای هوشمند امکان ایجاد تراکنش‌های معتبر بدون واسط را فراهم می‌کنند. این تراکنش‌ها قابل‌پیگیری و غیرقابل‌برگشت هستند. این قراردادها می‌توانند بدون نیاز به فرد یا نهادی اجرا و اعمال شود و از این رو می‌توانند امنیت بیشتر و هزینه کمتری داشته باشد.
 
 اتریم معروف‌ترین پلتفرم قراردادهای هوشمند است که به صورت متن‌باز بر بستر زنجیره‌ی بلوکی و با رایانش توزیع‌شده فعالیت می‌کند. این پلتفرم ماشین مجازی اتریوم 
 \LTRfootnote{EVM}
 را فراهم می‌آورد , اسکریپت‌ها را با استفاده از شبکهٔ بین‌المللی گره‌های عمومی اجرا می‌کند. اتریوم هم چنین یک توکن ارز رمزپایه به نام اتر
\LTRfootnote{Ether}
  ارائه می‌کند، که بین حساب‌ها قابل‌انتقال بوده و می‌تواند به گره‌های شرکت‌کننده برای محاسباتی که انجام داده‌اند پرداخت شود. گاز، سازوکار کار داخلی قیمت‌گذاری است که برای مکان‌یابی منابع روی شبکه به کار گرفته می‌شود. 
 \cite{radziwill}

\subsection{مقیاس‌پذیری}
در تمامی سیستم‌های کامپیوتری قابلیت افزایش و گسترش‌پذیری گره‌های محاسباتی می‌بایست وجود داشته باشد. در سیستم‌های توزیع‌شده هر کدام از گره های محاسباتی یک سیستم کامپیوتری منفک هستند. قابلیت گسترش‌پذیری این امکان را در اختیار سیستم می‌گذارد تا گره‌های محاسباتی بیشتری به سیستم اضافه شود. در زنجیره‌ی بلوکی از آن‌جایی که تمامی گره‌ها باید تمامی محتویات زنجیره را بر روی خود ذخیره کنند، امکان اضافه‌شدن یک گره جدید به سیستم با وجود این محدودیت بسیار صعب و سخت است. 

امروزه با افزایش تعداد گره‌های کاونده و افزایش حجم زنجیره‌ی بلوکی که خود معلول افزایش میزان تراکنش‌ها و هم‌چنین افزایش حجم معاملات است، چالش مقیاس‌پذیری در این تکنولوژی نوظهورنمود بیشتری داشته‌است. البته به منظور حل چالش مقیاس‌پذیری در زنجیره‌ی بلوکی، راهکارهای متعددی ارائه شده‌است. برخی از این راهکارها با ذات و طبیعت زنجیره‌ی بلوکی که از یک شبکه‌ی همتابه‌همتا استفاده می‌کند، متناقض هستند. این راهکارها شبکه‌ی مورد استفاده را به سمت متمرکزشدن
\LTRfootnote{entralized}
سوق می‌دهد.

\subsection{فشرده‌سازی}
فشرده‌سازی به معنای کدگذاری یک داده به نحوی‌ست که نسبت به داده‌ی اصلی تعداد بیت‌های کمتری اشغال کند. مبنای عملیات فشرده‌سازی وجود افزونگی
\LTRfootnote{redundancy}
 در اطلاعات است. فشرده‌سازی موجب صرفه‌جویی در فضای اشغال‌شده توسط داده می‌شود. فرایند‌های فشرده‌سازی و بازیابی داده‌ی اصلی از اطلاعات فشرده‌شده منابع محاسباتی مصرف می‌کنند. به همین دلیل فشرده‌سازی معاوضه بین فضای دخیره‌سازی و زمان پردازش است. دو معیار مهم در فشرده‌سازی نرخ فشرده‌سازی
 \LTRfootnote{compression ratio}
  و فضای صرفه‌جویی‌شده
  \LTRfootnote{space saving}
  هستند. 

%%%%%%%%%%%%%%%%%%%%% CONSLUSION %%%%%%%%%%%%%%%%%%%%%
\section{جمع‌بندی}
