% !TeX root=../main.tex
\chapter{بحث و نتیجه‌گیری}
%\thispagestyle{empty} 
\section{جمع‌بندی}
در این پژوهش در گام نخست با زمینه‌ی تحقیق، اهداف پیگیری‌شده و کارهایی که سابقاً در این زمینه انجام شده‌است آشنا شدیم. سپس مفاهیم پایه‌ای مربوط به آزمایش‌های انجام‌شده عنوان و بررسی شد. در ادامه راه پیشنهادی ارائه شد و ابزارهایی که در این مسیر مورد استفاده قرار گرفتند معرفی شدند. به جهت دستیابی به نتایج قابل‌تفسیر معیارهایی برای ارزیابی مطرح شدند. در نهایت با اعلان و تفسیر نتایج به‌دست‌آمده، مقایسه‌ای بین روش‌های مختلف فشرده‌سازی با تمرکز بر قراردادهای هوشمند بر بستر زنجیره‌ی بلوکی انجام شد. روش‌های فشرده‌سازی مطرح‌شده که در دو حالت منفرد و دسته‌ای مورد استفاده قرار گرفته‌اند، می‌توانند با توجه به امکانات و خواسته‌های پروژه، برای حل چالش مقیاس‌پذیری به کار برده شوند.

\section{نتیجه‌گیری}
با توجه به معیارهای ارزیابی و نتایج به‌دست‌آمده، دریافتیم که انتخاب روش مناسب فشرده‌سازی برای حل چالش مقیاس‌پذیری به هدف پرو‌ژه و منابع موجود بستگی دارد. هم‌چنین از دو حالت منفرد و دسته‌ای می‌توان استفاده نمود که هریک کاستی‌ها و برتری‌های خاص خود را دارند. همواره بین میزان فضای صرفه‌جویی‌شده که مستقیماً بر مقیاس‌پذیری تأثیرگذار است و توان مصرفی و سربار زمانی یک معامله وجود دارد که می‌بایست با دقت بررسی شود.

\section{دست‌آوردها}
از آن‌جایی که مقیاس‌پذیری یکی از اصلی‌ترین چالش‌های امروزه‌ی قراردادهای هوشمند بر پایه‌ی زنجیره‌ی بلوکی است، پژوهش‌های انجام‌شده در این زمینه از اهمیت بالایی برخوردار هستند. در این پژوهش یک روش برای برطرف کردن این چالش، یعنی فشرده‌سازی، مطرح شد. با توجه به نتایج به‌دست‌آمده مشاهده شد که این روش می‌تواند تا حد قابل‌قبولی از حجم اطلاعات ذخیره‌شده توسط زنجیره‌ی بلوکی بکاهد و با کاهش میزان داده‌ی نگهداری‌شده چالش مقیاس‌پذیری را مرتفع سازد. وجود چالش مقیاس‌پذیری مانع از رشد برنامه‌های پیاده‌سازی‌شده بر پایه‌ی زنجیره‌ی بلوکی می‌شود و با رفع‌نمودن آن می‌توان هرچه بیشتر به امکانات ارائه شد توسط این تکنولوژی نوین دست یافت.

\section{محدودیت‌ها}
از آن‌جایی که نحوه‌ی زمان‌بندی سیستم عامل توسط ما قابل کنترل نیست، محاسبه‌ی توان مصرفی با دقت بالا انجام نمی‌شود. به همین جهت لازم است تنها با خواندن مقادیر رجیسترهای مربوطه توان مصرفی پردازنده را تخمین زد که مقدار به‌دست‌آمده با مقدار واقعی تفاوت خواهد داشت.

\section{پیشنهادها}
در ادامه‌ی این پژوهش لازم است روند فشرده‌سازی به طور خودکار بر داده‌های ذخیره‌شده در زنجیره‌ی بلوکی اعمال شود و هم‌چنین مکانیزمی طراحی شود که در لزوم دسترسی به داده‌های پیشین این داده‌های فشرده‌سازی‌شده مجدداً بسط داده شده و به حالت اولیه بازگردانده شوند تا اطلاعات مربوطه خوانده شوند.