% !TeX root=../main.tex
\chapter{روش تحقیق}

%%%%%%%%%%%%%%%%%%%%% INTRO %%%%%%%%%%%%%%%%%%%%%
\section{مقدمه} 
در این فصل نخست روش پیشنهادی خود برای بررسی فشرده‌سازی اطلاعات قراردادهای هوشمند ‌بر بستر زنجیره‌ی بلوکی را معرفی می‌کنیم. در روند انجام این پژوهش از ابزارهای متفاوتی در مراحل پیاده‌سازی، آزمون و اندازه‌گیری معیارها استفاده شده‌است که در ادامه معرفی خواهند شد. هم‌چنین برای مقایسه بین روش‌های متفاوت فشرده‌سازی، نیاز است معیارهایی ارائه شوند. معیارهای مورد نظر در این پژوهش میزان فضای صرفه‌جویی‌شده، توان مصرفی و سربار زمانی هستند. 

%%%%%%%%%%%%%%%%%%%%% METHOD %%%%%%%%%%%%%%%%%%%%%
\section{روش پیشنهادی}
همان‌طور که اشاره شد، روش پیشنهادی یک روش مبتنی بر فشرده‌سازی و بازنگری در ساختار زنجیره‌ی بلوکی است. به همین منظور، روش پیشنهادی سه فاز پیاده‌سازی خواهد داشت. در فاز اول یک زنجیره‌ی بلوکی با حجم بالا مبتنی بر زنجیره‌ی بلوکی اتریوم پیاده‌سازی خواهد شد. در گام بعدی، ‌روش‌های مختلف فشرده‌سازی که بر اساس رخ‌دادن پیشامد هر کدام از سمبل‌ها به آن‌ها یک رشته اختصاص می‌دهد،‌ پیاده‌سازی خواهد شد. در گام نهایی، می بایست روش‌های مختلف فشرده‌سازی بر زنجیره‌ی بلوکی پیاده‌سازی‌شده اعمال گردد. این روش، فرایند مقیاس‌پذیری را با رویکرد بهبود ذخیره‌سازی تسهیل خواهد کرد.

می‌دانیم بر اساس اصول موجود در تئوری اطلاعات و اصول آنتروپی کدینگ‌ هرچه میزان شباهت در داده‌ها بیشتر باشد،‌ طول بیتی که به سمبل مورد نظر اختصاص خواهد یافت کوتاه‌تر می‌شود. به منظور افزایش بهره‌وری فشرده‌سازی این فرایند در دو حالت بلوک‌های منفرد و بلوک‌های دسته‌ای مورد آزمایش قرار گرفت. پیش‌بینی می‌شود‌ میزان فضای صرفه‌جویی‌شده در حالت فشرده‌سازی دسته‌ای به مراتب بیشتر از حالت منفرد باشد. دلیل پیشنهادشدن روش فشرده‌سازی برای مقابله چالش مقیاس‌پذیری، ذات اطلاعات موجود در قراردادهای هوشمند است. در قراردادهای هوشمند اطلاعات ذخیره‌شده بایت‌کد مربوط به منطق برنامه است. در نتیجه می‌توان توقع داشت که بین بلوک‌ها افزونگی داشته باشیم.

مقایسه‌ی حجم اشغال‌شده توسط زنجیره‌ی بلوکی اصلی که در فاز یک پیاده‌سازی شده‌است،‌ با حجم اشغال‌شده پس از گام سوم می‌تواند کلیدی‌ترین معیار ارزیابی قلمداد شود. از سوی دیگر، بدیهی‌ست اعمال هر فرایند فشرده‌سازی نیازمند صرف زمان و توان مصرفی‌ست. برای هرچه دقیق‌تر شدن محاسبات،‌ مقدار فضای صرفه‌جویی‌شده، توان مصرفی و سربار زمانی هر اجرا گزارش خواهد شد. هر کدام از پارامترهای یادشده می‌تواند به عنوان به پارامتر اصلی ورود به مسئله مورد مقایسه قرار گیرند.

قرارداد هوشمند پیاده‌شده، یک فهرست کارها
\LTRfootnote{Todo List}
است. برای افزایش حجم زنجیره‌ی بلوکی، پنجاه‌هزار بلوک تولید شده‌است. هر بلوک نشان‌دهنده‌ی اضافه‌کردن یک کار به لیست است. از آن‌جایی که هر یک از کارها به صورت یک رشته نگهداری می‌شوند، این بلوک‌های رشته‌هایی به طول ۲۰ کاراکتر را به لیست اضافه می‌کنند. برای جلوگیری از افزونگی زیاد بین بلوک‌ها، هر یک از این رشته‌ها به طور تصادفی تولید می‌شوند و سپس به زنجیره اضافه می‌شوند.

%%%%%%%%%%%%%%%%%%%%% TOOLS %%%%%%%%%%%%%%%%%%%%%
\section{ابزار}
برای پیاده‌سازی قرارداد هوشمند مورد نظر به جهت اعمال و بررسی الگوریتم‌های فشرده‌سازی به ابزار مرتبط با این تکنولوژی نیازمندیم. در ابتدا بستر زنجیره‌ی بلوکی را برای پیاده‌سازی قرارداد هوشمند فراهم می‌کنیم. سپس نیاز است تا یک برنامه‌ی نمونه بنویسیم و آن را به صورت قرارداد هوشمند اجرا کنیم. از آن‌جایی که الگوریتم‌های فشرده‌سازی می‌بایست بر روی داده‌های ذخیره‌شده در زنجیره‌ی بلوکی اعمال شوند، لازم است مقدار قابل‌توجهی داده تولید کنیم. در این مسیر از ابزارهایی استفاده شده‌است که شرح داده خواهندشد.

\subsection{Solidity}
سالیدیتی یک زبان شئ‌گرا برای نوشتن قراردادهای هوشمند است. از این زبان بر بستر زنجیره‌های بلوکی و به طور خاص اتریوم استفاده می‌شود. از آن‌جایی که این زبان مهم‌ترین زبان مورداستفاده برای اتریوم است، تطابق کامل برای اجرا روی ماشین مجازی اتریوم(\lr{EVM}) دارد. برنامه‌ی کامپیوتری نوشته‌شده با سالیدیتی به بایت‌کدی تبدیل می‌شود که قابل‌اجرا بر روی ماشین مجازی است. با استفاده از این زبان برنامه‌نویسان قادرند منطق بیزنسی برنامه‌ی مورد نظر را پیاده‌سازی کنند و این امکان فراهم می‌شود که معاملات و تراکنش‌های انجام‌شده به صورت غیرقابل‌ویرایش و غیرقابل‌برگشت ضبط و نگهداری شوند. 

\subsection{\lr{Truffle Suite}}
این ابزار امکانات لازم برای پیاده‌سازی برنامه، تست‌کردن آن و ایجاد خط لوله
\LTRfootnote{Pipeline}
بر بستر اتریوم را فراهم می‌آورد. هدف اصلی این ابزار ایجاد سهولت در روند برنامه‌نویسی برای اتریوم است. از قابلیت‌های آن می‌توان به توانایی کامپایل و لینک کردن قراردادهای هوشمند و مدیریت باینری‌های مربوط به آن اشاره کرد. هم‌چنین با استفاده از خط لوله‌های قابل شخصی‌سازی، می‌توان روند ساخت، آزمون و برپایی برنامه را تعیین کرد. از آن‌جایی که این ابزار بر بستر اتریوم فعالیت می‌کند، توانایی اتصال و کارکردن با شبکه‌های عمومی و شخصی و زنجیره‌ی بلوکی اتریوم را داراست. 

\subsection{Ganache}
از آن‌جایی که برای تست و بررسی صحت عملکرد قراردادهای هوشمند نوشته‌شده نمی‌توانیم از زنجیره‌ی بلوکی عمومی اتریوم استفاده کنیم، نیاز است تا از یک زنجیره‌ی بلوکی خصوصی بهره بگیریم. هم‌چنین از آن‌جایی که هر عمل بر روی زنجیره‌ی بلوکی اتریم مقداری اتر مصرف می‌کند، لازم است از تعدادی کیف‌پول نمونه برای تست‌کردن استفاده شود. ابزار Ganache امکان ایجاد یک زنجیره‌ی بلوکی اتریوم شخصی برای برپاسازی قرارداد هوشمند نوشته و اجراکردن تست‌های مربوط به آن را فراهم می‌آورد. با استفاده از این ابزار می‌توان وضعیت زنجیره‌ی بلوکی را در هر لحظه مشاهده و مدیریت کرد و هم‌چنین می‌توان با استفاده از دستورهای مربوطه، عملیات مشخصی روی زنجیره‌ی بلوکی انجام داد.  

\subsection{\lr{Intel Power Gadget}}
ابزار \lr{Power Gadget} یک نرم‌افزار ارائه‌شده توسط شرکت Intel است که برای بررسی توان و انرژی مصرفی توسط پردازنده‌های این شرکت قابل‌استفاده است. از آن‌جایی که روش‌های سنتی تخمین و محسابه‌ی توان و انرژی پردازنده پیچیده و نیازمند بهره‌گیری از اماکانات جانبی است، این ابزار کمک زیادی در راستای سهولت بخشیدن به این محسبات انجام داده‌است. هم‌چنین با استفاده از این ابزار می‌توان خصوصیات دیگر سیستم از جمله فرکانس کاری پردازنده، دما و میزان بهره‌وری آن را مشاهده و مدیریت کرد.

%%%%%%%%%%%%%%%%%%%%% EVALUATION %%%%%%%%%%%%%%%%%%%%%
\section{معیارهای ارزیابی}
با توجه به هدف پژوهش، سه معیار ارزیابی فضای صرفه‌جویی‌شده، توان مصرفی و سربار زمانی در نظر گرفته شده‌اند.

\subsection{فضای صرفه‌جویی‌شده}
مهم‌ترین معیار مقایسه برای بررسی میزان اثرگذاری الگوریتم فشرده‌سازی، مقایسه‌ی حجم داده پیش و پس از فشرده‌سازی است. فضای صرفه‌جویی‌شده به صورت میزان فضایی که در داده‌ی فشرده‌شده کم‌تر از داده‌ی اصلی اشغال شده‌است، تعریف می‌شود. برای یکسان‌سازی این معیار در بین داده‌های با اندازه‌های متفاوت، تفاوت دو داده را بر سایز داده‌ی اصلی تقسیم می‌کنیم و آن را به صورت درصد در نظر می‌گیریم. در نتیجه درصد فضای صرفه‌جویی‌شده از رابطه‌ی زیر به دست می‌آید که در آن 
${D_o}$
 اندازه‌ی داده‌ی اصلی و 
${D_c}$
اندازه‌ی داده‌ی فشرده‌شده است.
\[ \frac{D_o - D_c}{D_o}\]

\subsection{توان مصرفی}
اعمال الگوریتم فشرده‌سازی باعث مصرف‌شدن توان در پردازنده می‌شود. با توجه به هزینه‌ی ناشی از مصرف توان،‌ واضح است که این مورد یکی از معیارهای قابل‌توجه برای مقایسه‌ی الگوریتم‌ها است. 

\subsection{سربار زمانی}
فشرده‌سازی داده‌ها به علت اجرای الگوریتم مربوطه باعث صرف زمان بیشتری می‌شود و نسبت به حالت بدون فشرده‌سازی، یکی سربار زمانی ایجاد می‌کند. این سربار زمانی با توجه به کاربرد مورد نظر می‌تواند معیار تعیین‌کننده‌ای باشد.

%%%%%%%%%%%%%%%%%%%%% CONSLUSION %%%%%%%%%%%%%%%%%%%%%
\section{‌جمع‌بندی}
با توجه به هدف پژوهش، روشی با سه فاز برای انجام آن پیشنهاد شده‌است. در فاز اول با استفاده از ابزارهای توصیف‌شده به پیاده‌سازی یک قرارداد هوشمند بر پایه‌ی زنجیره‌ی بلوکی اتریوم پرداختیم. سپس با انجام عملیات بر روی قرارداد هوشمند ایجادشده، داده‌ای با حجم سنگین تولید شد. در ادامه روش‌های مورد نظر برای فشرده‌سازی به طور مستقل پیاده‌سازی شدند. برای بررسی تأثیر روش‌‌های فشرده‌سازی بر روی اطلاعات مربوط به قرارداد هوشمند پیاده‌سازی‌شده بر روی زنجیره‌ی بلوکی، این روش‌های فشرده‌سازی بر اطلاعات ایجاد‌شده اعمال شد و معیارهای فضای ذخیره‌شده، توان مصرفی و سربار زمانی درباره‌ی هریک از روش‌ها بررسی شد.