% !TeX root=../main.tex
% در این فایل، عنوان پایان‌نامه، مشخصات خود و چکیده پایان‌نامه را به انگلیسی، وارد کنید.

%%%%%%%%%%%%%%%%%%%%%%%%%%%%%%%%%%%%
\latinuniversity{University of Tehran}
\latincollege{College of Engineering}
\latinfaculty{Faculty of Electrical and Computer Engineering}
\latindepartment{Computer Architecture}
\latinsubject{Computer Engineering}
\latinfield{Computer Architecture}
\latintitle{Implementating Smart Contracts on Blockchain Platform with an Approach to Improve Scalability}
\firstlatinsupervisor{Dr. Saeed Safari}
%\secondlatinsupervisor{Second Supervisor}
%\firstlatinadvisor{First Advisor}
%\secondlatinadvisor{Second Advisor}
\latinname{Armita}
\latinsurname{Safa}
\latinthesisdate{January 2020}
\latinkeywords{Blockchain, Smart Contracts, Scalability, Compression, Ethereum}
\en-abstract{
	Today, blockchain is one of the evolving technologies whose various applications have made it prevalent. One of the most important usages of blockchain can be seen in Smart Contracts. Considering the high volume of transactions carried out on the peer-to-peer blockchain network, scalability is believed to be one of the biggest challenges in this domain. Currently, some of the most commonly used blockchains have more than 100GB of data stored on them. When a miner joins the network, it must download the whole chain before attempting to start the computation process. This continuously increasing size of data is a major issue while developing and using blockchain and therefore limits the growth of applications based on blockchain. In order to overcome the scalability challenge, different solutions have been proposed, some of which focus on changing the nature of blockchain and its composing elements. Other solutions aim to modify data storage. In Smart Contracts, the stored data is the bytecode that provides the program's logic. Hence, compression techniques can be used to reduce the size of data. In this project, different compression methods are applied to a sample high volume blockchain. Those methods are compared with regard to space saving, time overhead and processor power.
}
