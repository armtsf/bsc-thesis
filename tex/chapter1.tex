% !TeX root=../main.tex

\chapter{مقدمه و بیان مسئله}
% دستور زیر باعث عدم‌نمایش شماره صفحه در اولین صفحه‌ی این فصل می‌شود.
%%%%%%%%%%%%%%%%%%%%% INTRO %%%%%%%%%%%%%%%%%%%%%
%\thispagestyle{empty}
\section{مقدمه}
ذخیره‌ی ایمن و مطمئن اطلاعات همواره یکی از چالش‌های اساسی در حوزه سیستم‌های کامپیوتری بوده‌است. امروزه با افزایش روزافزون و چشم‌گیر استفاده از سیستم‌های کامپیوتری و همچنین ظهور و بروز مفاهیمی همچون کلان‌داده این نیاز بیشتر حس می‌شود. به عنوان اولین راهکار می‌توان از توزیع‌شدگی نام‌برد. استفاده از یک پایگاه‌داده ایمن و مطمئن به منظور ذخیره‌سازی اطلاعات می‌تواند یک راهکار مناسب برای پاسخگویی به این نیاز برشمرده شود.

\subsection{زنجیره‌ی بلوکی}
با وجود اینکه، استفاده از یک پایگاه‌داده توزیع‌شده مناسب به نظر می‌رسد، اما در صورت اختلال در کارکرد هرکدام از گره‌های ذخیره‌سازی، روند دسترسی به داده‌ها با مشکل روبرو خواهدشد. در این میان برای حل این مشکل از افزونگی و تکنیک‌های موجود در ذخیره‌سازی استفاده می‌کنند. بدیهی‌است در این میان هرچه میزان افزونگی بیشتر شود، امکان بازیابی موثر اطلاعات افزایش یافته ولی سربار سیستم افزایش خواهد یافت. استفاده از یک شبکه همتا‌به‌همتا که در آن تمامی گره‌های عضو شبکه امکان اشتراک‌گذاری و استفاده از منابع یکدیگر را داشته باشند، می‌تواند به عنوان راهکار برای این چالش در نظر گرفته‌شود.


با استفاده از تمامی مفاهیم فوق در سال ۲۰۰۹ برای نخستین بار از یک مفهوم به نام زنجیره‌ی بلوکی رونمایی شد، که در آن داده‌ها در یک شبکه‌ی همتا‌به‌همتا ذخیره‌ می‌شدند. به منظور تامین امنیت ذخیره‌سازی داده‌ها، از فرآیند‌های مرتبط با رمزنگاری استفاده شده‌است. به عنوان برنامه‌ی کاربردی مرتبط با این پایگاه‌داده، از یک رمزارز استفاده‌شد. زنجیره‌ی بلوکی این قابلیت را در اختیار کاربران قرار داد تا با استفاده از آن بسترهای متعددی از قبیل رمزارز و قراردادهای هوشمند نیز ظهور و توسعه پیدا‌کنند.
\subsection{چالش‌های زنجیره‌ی بلوکی}
بدیهی است هر کدام از تکنولوژی‌های نوظهور با چالش‌هایی روبرو خواهندبود. زنجیره‌ی بلوکی نیز با چالش‌های جدی و اساسی‌ای روبرو است. چالش‌هایی از قبیل مقیاس‌پذیری، نشت اطلاعات و هزینه بالای سرویس به عنوان چالش‌های اساسی در این حوزه مطرح می‌شوند. برای هرکدام از چالش‌های نام‌برده شده راه‌حل‌های زیادی نام‌برده شده‌است. هریک از این راهکارها منتج به ایجاد شاخه‌های دیگری از زنجیره‌های بلوکی شده‌است.


در زنجیره‌بلوکی تمامی اطلاعات و داده‌ها باید در این بستر ذخیره شوند. با فراگیر شدن این پدیده نوظهور و اعمال آن برروی کاربردهایی از قبیل رمزارز و قراردادهای هوشمند، تمایل استفاده از این فناوری نوظهور به شدت افزایش یافته‌است. افزایش تعداد تراکنش‌ها در این بستر و همچنین افزایش روزافزون حجم معاملات در این بستر سبب شده تا این فناوری با چالش مقیاس‌پذیری بیشتر از سایر چالش‌ها روبرو باشد. در این پایان‌نامه سعی شده‌است تا چالش مقیاس‌پذیری با رویکرد بهبود روند ذخیره‌سازی بر روی ذخیره‌بلوکی اتریوم بهبود یابد.

%%%%%%%%%%%%%%%%%%%%% BACKGROUND %%%%%%%%%%%%%%%%%%%%%
\section{تاریخچه‌ای از موضوع تحقیق}
همان‌‌طور که اشاره شد، تمامی تراکنش‌های بر بستر زنجیره‌ی بلوکی می‌بایست در این زنجیره ذخیره شوند. با توجه به افزایش روزافزون تراکنش‌ها و معاملات صورت‌گرفته، حجم اطلاعات موردنیاز برای ذخیره‌سازی بر این بستر افزایش چشم‌گیری داشته است. این افزایش موجب به‌وجود‌آمدن چالش‌هایی از قبیل مقیاس‌پذیری، نشت اطلاعات و \dots شده‌است. در \cite{Lin2017}
تمامی چالش‌های موجود در این زمینه نام برده شده و در میان آن‌ها مقیاس‌پذیری به عنوان بزرگ‌ترین چالش پیش‌ روی زنجیر‌ه‌ی بلوکی شناخته شده‌است. 

در حال حاضر بسیاری از زنجیره‌های بلوکی مطرح، حجمی بالای ۱۰۰ گیگابایت را شامل می‌شوند. هنگامی که یک گره کاونده به شبکه زنجیره‌ی بلوکی اضافه می‌شود، ابتدا تمامی زنجیره‌ی قبلی را باربرداری و سپس شروع به حل فرآیند پیچیده‌ی محاسباتی می‌کند. این حجم بسیار بالا که به طور روزمره نیز مقدار آن افزوده می‌شود، یک چالش اساسی هنگام استفاده از زنجیره‌ی بلوکی است. \cite{Wang2018}
به منظور حل مشکل مقیاس‌پذیری موجود در زنجیره‌ی بلوکی، راه‌کارهای بسیاری از قبیل بهبود و بهینه‌سازی روند ذخیره‌سازی مطرح شده‌است.
در \cite{Kim2018} تمامی راه‌کارهای مطرح‌شده در حوزه‌ی مقیاس‌پذیری به پنج دسته‌ی کلی طبقه‌بندی شده‌اند؛ راه‌کارهایی که به اعمال تغییر بر روی عناصر زنجیره‌ی اصلی می‌پردازند، راه‌کارهایی با پردازش تراکنش‌ها در خارج از زنجیره‌ی اصلی، راه‌کارهایی با کمک‌گرفتن از زنجیره‌ی جانبی، راه‌کارهایی با ایجاد ساختار والد-مولد و راه‌کارهایی با فراهم کردن امکان ارتباط بین زنجیره‌های متعدد.
در ادامه به بررسی تعدادی از راه‌کارهای متداول استفاده‌شده در این حوزه می‌پردازیم.

\subsection{راه‌کارهای مبتنی بر بازطراحی زنجیره‌ی بلوکی}
هدف اصلی در این دسته از راه‌کارها تغییر ساختار زنجیره‌ی بلوکی و گره‌های آن به نحوی‌ست که نگه‌داری اطلاعات بهینه شود. بهینه‌سازی اطلاعات به منظور کاهش حجم موردنیاز برای ذخیره‌سازی هر کدام از داده‌ها معنا می‌شود. مهم‌ترین راه‌کار ارائه‌شده در این دسته، \cite{Eyal2015} است. در این روش، بلوک‌های سنتی به دو دسته‌ی بلوک‌های کلیدی برای انتخاب رهبر و ریزبلوک‌ها برای ذخیره‌سازی تراکنش‌ها تقسیم می‌شوند. در این طرحواره، عناصر کاونده سعی در کسب رهبری دارند. بلوک رهبر وظیفه‌ی تولید ریزبلاک‌ها را تا زمان انتخاب رهبر جدید بر عهده دارد. به این ترتیب زنجیره‌ی بلوکی بازطراحی شده و حجم ذخیره‌سازی در دست برای هر بلاک افزایش می‌یابد. 

\subsection{راه‌کارهای مبتنی بر اصلاح معیارهای ذخیره‌سازی}
این دسته از راهکارها، مشتمل بر روش‌هایی هستند که با بازتعریف‌کردن معیارهایی برای عناصر زنجیره‌‌ی بلوکی لزوم نگهداری هر بلوک برای گره‌های مختلف زنجیره را تعیین می‌نمایند. در راه‌کار ارائه‌شده در \cite{Bruce2017} تراکنش‌های قدیمی حذف شده و اعتبار آدرس‌ها در درخت حساب‌ها نگهداری می‌شود. به این ترتیب گره‌های شبکه لازم نیست برای بررسی درستی بلوک‌ها تمامی بلوک‌های زنجیره را نگهداری کنند. این روش به کاهش چشم‌گیر حجم اطلاعات دریافتی توسط گره‌های شبکه می‌انجامد. 

در راه‌کار دیگری که ذیل همین گروه بیان شده‌است، ‌می‌توان به \cite{VanDenHooff2014} اشاره کرد. این روش نیز بر ایجاد گره‌های سبک در شبکه تأکید دارد. ایده‌ی کلی این روش بر پایه‌ی برون‌سپاری محاسبات از طرف گره‌های سبک است. به منظور اطمینان از صحت نتایج محاسبات انجام‌شده، نتایج به‌دست‌آمده از کارگزارهای متفاوت با هم مقایسه می‌شوند.

\subsection{راه‌کارهای مبتنی بر بهبود ذخیره‌سازی}
راه‌کارهای ارائه‌شده در این دسته به طور مستقیم بر نحوه‌ی ذخیره‌سازی اطلاعات زنجیره‌ی بلوکی تمرکز دارند. در \cite{Pontiveros2018} یک روش فشرده‌سازی قراردادهای هوشمند بر بستر اتریوم پیشنهاد شده‌است. روش استفاده‌شده از تشابه کد اجرایی استفاده‌شده توسط قراردادهای هوشمند نگه‌داری‌شده در زنجیره بهره می‌گیرد. به این ترتیب با فشرده‌سازی اطلاعات، از حجم آن‌ها کاسته شده و در فضای ذخیره‌سازی صرفه‌جویی می‌شود.

%%%%%%%%%%%%%%%%%%%%% EXPLANATION %%%%%%%%%%%%%%%%%%%%%
\section{شرح مسئله تحقیق}
تحقیقات گسترده‌ای در حوزه حل چالش مقیاس‌پذیری در زنجیره‌ی بلوکی انجام شده‌است. در این پایان‌نامه سعی در کاهش حجم مصرفی برای ذخیره‌سازی زنجیره‌ی بلوکی داریم. می‌دانیم در تمامی زنجیره‌های بلوکی تمامی داده و رکوردها باید از ابتدا ذخیره شوند، همین امر سبب می‌شود تا حجم لازم برای ذخیره‌سازی این قسم پایگاه‌داده‌ها به شدت افزایش یابد. همان‌گونه که در بخش قبل اشاره‌شد، راهکارهای زیادی در این زمینه ارائه شده‌است. راهکار پیشنهادی از آنجا که به طور مستقیم بر نحوه ذخیره‌سازی تمرکز دارد، در گروه راهکارهای مبتنی بر بهبود ذخیره‌سازی تقسیم‌یندی می‌شود. 

%%%%%%%%%%%%%%%%%%%%% DESCRIPTION %%%%%%%%%%%%%%%%%%%%%
\section{تعریف موضوع تحقیق}
همان‌طور که اشاره‌شد، روش‌های بسیاری در حوزه حل مشکل‌ مقیاس‌پذیری زنجیره‌ی بلوکی انجام شده‌اند. در این پایان‌نامه راهکار ارائه شده به طور مشخص در گروه راهکارهای مبتنی بر بهبود ذخیره‌سازی قرار می‌گیرد. در روش ارائه شده مشکل مقیاس‌پذیری با استفاده از بهینه‌ ساختن روش ذخیره‌سازی انجام می‌شود. در روش ارائه شده تمامی بلوک‌های موجود در زنجیره‌ی بلوکی به نحوی فشرده شده و بلوک‌های فشرده‌شده با بلوک‌های اصلی جایگزین می‌شوند. می‌توان روش ارائه شده را یک روش بهبود مقیاس‌پذیری مبتنی بر فشرده‌سازی نیز بیان کرد.


در روش ارئه شده تمامی محتویات ذخیره‌شده در زنجیره‌های بلوکی فشرده‌شده فلذا حجم مورد نیاز برای ذخیره‌سازی این زنجیره کاهش خواهد یافت. بدیهی‌است انجام فرآیند ذخیره‌سازی با سربارهای زمانی روبرو خواهد بود. همچنین فرآیند فشرده‌سازی نیازمند صرف توان مصرفی و  اعمال بار پردازشی بر سیستم‌ خواهد بود. ذخیره‌سازی باید به صورتی باشد که میزان فشرده‌سازی توجیه سرباز زمانی و پردازشی سیستم‌ را بکند.

%%%%%%%%%%%%%%%%%%%%% GOALS %%%%%%%%%%%%%%%%%%%%%
\section{اهداف و آرمان‌های کلی تحقیق}
در روش ارائه‌شده، مشکل مقیاس‌پذیری با استفاده از روش‌های فشرده‌سازی مرتفع خواهدگردید. از آنجاییکه روش مورد نظر به صورت کاملا دقیق و جزئی به روش‌های فشرده‌سازی وابسته‌ خواهد‌بود، پس یک کاوش در حوزه روش‌های فشرده‌سازی می‌بایست انجام شود. روش‌های فشرده‌سازی به طور کلی برمبنای احتمال وقوع هرکدام از سمبل‌ها، یک عبارت را در نظر می‌گیرند. عبارت مورد نظر به طوریست که هرجه احتمال وقوع یک سمبل بیشتر باشد، آنگاه طول عبارت اختصاص یافته کوتاه‌تر خواهد بود. 


اکنون با در نظر گرفتن کاوش‌های انجام شده و بدست آوردن روش مورد نظر برای ذخیره‌سازی، می‌بایست الگوریتم ذخیره‌سازی را بر زنجیره‌بلوکی اعمال کرد. همان‌طور که اشاره‌شد، زنجیره‌ی بلوکی ذاتا از یک شبکه توزیع‌شده‌ی همتا‌به‌همتا استفاده می‌کند، روش ذخیره‌سازی به هیچ عنوان نباید متناقض با ذات زنجیره‌ی بلوکی باشد. این بدین معناست که روش ذخیره‌سازی ارائه شده به هیچ عنوان نباید به نحوی‌ باشد که زنحیره را به سمت یک شبکه متمرکز سوق دهد.

%%%%%%%%%%%%%%%%%%%%% METHOD %%%%%%%%%%%%%%%%%%%%%
\section{روش انجام تحقیق}
زنجیره‌ی بلوکی را می‌توان به عنوان پایگاه‌داده برروی کاربردهای مختلفی بکار گرفت. یکی از مهمترین کاربردهای مورد استفاده از زنحیره‌ی بلوکی استفاده به عنوان پایگاه‌داده‌‌ای مطمئن برای حفظ و ذخیره‌سازی قراردا‌د‌های هوشمند است. در این پایان‌نامه به منظور ایجاد یک زنجیره‌ی بلوکی از زنجیره‌ی بلوکی مبتنی بر اتریوم استفاده ‌شده‌است.

فرآیند فشرده‌سازی همراه با سربار زمانی و صرف توان مصرفی خواهدبود. با در نظر گرفتن این امر می‌توان روش ارائه‌شده را از ابعاد مختلف مورد بررسی و واکاوی قرار داد. روش ارائه‌شده می‌تواند به صور مختلفی در نظر گرفته‌شود این بدین معناست که روش پیشنهادی می‌تواند با توجه به محدودیت‌های موجود و در نظر گرفته‌شده برای هر سیستم خود را تطبیق دهد.

ذیل برگزاری آزمایشات برای آزمودن روش پیشنهادی، پارامترهایی از قببل میزان فشرده‌سازی، سربار زمانی و همچنین توان مصرفی مورد آزمون قرار گرفته‌اند. روش پیشنهادی یک روش منعطف در برابر محدودیت‌ها است. با توجه به محدودیت‌های موجود در این بخش می‌تواند خود را تطبیق دهد.

%%%%%%%%%%%%%%%%%%%%% STRUCTURE %%%%%%%%%%%%%%%%%%%%%
\section{ساختار پایان‌نامه}