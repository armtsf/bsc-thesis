% !TeX root=../main.tex

\chapter{مقدمه و بیان مسئله}
% دستور زیر باعث عدم‌نمایش شماره صفحه در اولین صفحه‌ی این فصل می‌شود.
%\thispagestyle{empty}
\section{مقدمه}
بلاک‌چین،‌ ذخیره‌سازی، همه‌ی داده‌ها

\section{تاریخچه‌ای از موضوع تحقیق}
همان‌‌طور که اشاره شد، تمامی تراکنش‌های بر بستر زنجیره‌ی بلوکی می‌بایست در این زنجیره ذخیره شوند. با توجه به افزایش روزافزون تراکنش‌ها و معاملات صورت‌گرفته، حجم اطلاعات موردنیاز برای ذخیره‌سازی بر این بستر افزایش چشم‌گیری داشته است. این افزایش موجب به‌وجود‌آمدن چالش‌هایی از قبیل مقیاس‌پذیری، نشت اطلاعات و \dots شده‌است. در \cite{Lin2017}
تمامی چالش‌های موجود در این زمینه نام برده شده و در میان آن‌ها مقیاس‌پذیری به عنوان بزرگ‌ترین چالش پیش‌ روی زنجیر‌ه‌ی بلوکی شناخته شده‌است. 

در حال حاضر بسیاری از زنجیره‌های بلوکی مطرح، حجمی بالای ۱۰۰ گیگابایت را شامل می‌شوند. هنگامی که یک گره کاونده به شبکه زنجیره‌ی بلوکی اضافه می‌شود، ابتدا تمامی زنجیره‌ی قبلی را باربرداری و سپس شروع به حل فرآیند پیچیده‌ی محاسباتی می‌کند. این حجم بسیار بالا که به طور روزمره نیز مقدار آن افزوده می‌شود، یک چالش اساسی هنگام استفاده از زنجیره‌ی بلوکی است. \cite{Wang2018}
به منظور حل مشکل مقیاس‌پذیری موجود در زنجیره‌ی بلوکی، راه‌کارهای بسیاری از قبیل بهبود و بهینه‌سازی روند ذخیره‌سازی مطرح شده‌است.
در \cite{Kim2018} تمامی راه‌کارهای مطرح‌شده در حوزه‌ی مقیاس‌پذیری به پنج دسته‌ی کلی طبقه‌بندی شده‌اند؛ راه‌کارهایی که به اعمال تغییر بر روی عناصر زنجیره‌ی اصلی می‌پردازند، راه‌کارهایی با پردازش تراکنش‌ها در خارج از زنجیره‌ی اصلی، راه‌کارهایی با کمک‌گرفتن از زنجیره‌ی جانبی، راه‌کارهایی با ایجاد ساختار والد-مولد و راه‌کارهایی با فراهم کردن امکان ارتباط بین زنجیره‌های متعدد.
در ادامه به بررسی تعدادی از راه‌کارهای متداول استفاده‌شده در این حوزه می‌پردازیم.

\subsection{راه‌کارهای مبتنی بر بازطراحی زنجیره‌ی بلوکی}
هدف اصلی در این دسته از راه‌کارها تغییر ساختار زنجیره‌ی بلوکی و گره‌های آن به نحوی‌ست که نگه‌داری اطلاعات بهینه شود. بهینه‌سازی اطلاعات به منظور کاهش حجم موردنیاز برای ذخیره‌سازی هر کدام از داده‌ها معنا می‌شود. مهم‌ترین راه‌کار ارائه‌شده در این دسته، \cite{Eyal2015} است. در این روش، بلوک‌های سنتی به دو دسته‌ی بلوک‌های کلیدی برای انتخاب رهبر و ریزبلوک‌ها برای ذخیره‌سازی تراکنش‌ها تقسیم می‌شوند. در این طرحواره، عناصر کاونده سعی در کسب رهبری دارند. بلوک رهبر وظیفه‌ی تولید ریزبلاک‌ها را تا زمان انتخاب رهبر جدید بر عهده دارد. به این ترتیب زنجیره‌ی بلوکی بازطراحی شده و حجم ذخیره‌سازی در دست برای هر بلاک افزایش می‌یابد. 

\subsection{راه‌کارهای مبتنی بر اصلاح معیارهای ذخیره‌سازی}
این دسته از راهکارها، مشتمل بر روش‌هایی هستند که با بازتعریف‌کردن معیارهایی برای عناصر زنجیره‌‌ی بلوکی لزوم نگهداری هر بلوک برای گره‌های مختلف زنجیره را تعیین می‌نمایند. در راه‌کار ارائه‌شده در \cite{Bruce2017} تراکنش‌های قدیمی حذف شده و اعتبار آدرس‌ها در درخت حساب‌ها نگهداری می‌شود. به این ترتیب گره‌های شبکه لازم نیست برای بررسی درستی بلوک‌ها تمامی بلوک‌های زنجیره را نگهداری کنند. این روش به کاهش چشم‌گیر حجم اطلاعات دریافتی توسط گره‌های شبکه می‌انجامد. 

در راه‌کار دیگری که ذیل همین گروه بیان شده‌است، ‌می‌توان به \cite{VanDenHooff2014} اشاره کرد. این روش نیز بر ایجاد گره‌های سبک در شبکه تأکید دارد. ایده‌ی کلی این روش بر پایه‌ی برون‌سپاری محاسبات از طرف گره‌های سبک است. به منظور اطمینان از صحت نتایج محاسبات انجام‌شده، نتایج به‌دست‌آمده از کارگزارهای متفاوت با هم مقایسه می‌شوند.

\subsection{راه‌کارهای مبتنی بر بهبود ذخیره‌سازی}
راه‌کارهای ارائه‌شده در این دسته به طور مستقیم بر نحوه‌ی ذخیره‌سازی اطلاعات زنجیره‌ی بلوکی تمرکز دارند. در \cite{Pontiveros2018} یک روش فشرده‌سازی قراردادهای هوشمند بر بستر اتریوم پیشنهاد شده‌است. روش استفاده‌شده از تشابه کد اجرایی استفاده‌شده توسط قراردادهای هوشمند نگه‌داری‌شده در زنجیره بهره می‌گیرد. به این ترتیب با فشرده‌سازی اطلاعات، از حجم آن‌ها کاسته شده و در فضای ذخیره‌سازی صرفه‌جویی می‌شود.